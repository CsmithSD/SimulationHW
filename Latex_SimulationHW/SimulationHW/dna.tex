% !TEX root = Simulation.tex

\chapter{DNA Computing - Text Chapter 9}


\section{ Problem 1 }
\textbf{ Name four problems that cannot be solved by a Turing machine. } \\
The Halting Problem, Mortality Problem, Busy Beaver Champion, and Rice's Theorem are four problems that cannot be solved by a Turing Machine. Undecidable problems is the type of problem that a Turing machaine cannot solve. These requre a yes/no answer, however, there isn't a possible computer program that will always give a right answer. It would sometimes give the wrong answer or run forever without giving an answer. Descriptions of algorithms taken from wikipedia.
\begin{itemize}
\item The Halting Problem: Problem of determining, from a description of an arbitrary computer program and an input, whether the program will finish running or continue to run forever.
\item Mortality Problem: Given a Turing machine, decide whether it halts when run on any configuration(Not necessarily a starting one).
\item Busy Beaver Champion: A Turing machine that attains the maximum number of steps performed, or maximum number of nonblank symbols finally on the tape, among all Turing machines in a certain class. The Turing machines in this class must meet certain design specifications and are required to eventually halt after being started with a blank tape.It is undecidable by a general algorithm whether an arbitrary Turing machine is a busy beaver.
\item Rice's Theorem:for any non-trivial property of partial functions, there is no general and effective method to decide whether an algorithm computes a partial function with that property.
\end{itemize}

\section{ Problem 2 }
\textbf{ Name four NP-complete and four NP-hard problems. } \\
The Traveling Salesmen Problem (TSP), Himiltonian Path Problem, Satisfiability Problem for Propositional Formulas or Propositions (SAT) Problem, and the Knapsack Problem are four NP-complete problems. The Halting Problem, Cook's Theorem, Maximum Clique Size (from3SAT), and Vertex Cover (from Independent Set).
\begin{itemize}
\item TSP
\section{ Problem 5 }
\textbf{ The two most basic DNA sequencing techniques are known as a) Maxam-Gilbert and b) Sanger, after their proponents. Explain how each of these techniques work and contrast them. }

